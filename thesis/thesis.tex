% Options for packages loaded elsewhere
\PassOptionsToPackage{unicode}{hyperref}
\PassOptionsToPackage{hyphens}{url}
%
\documentclass[
]{article}
\usepackage{amsmath,amssymb}
\usepackage{iftex}
\ifPDFTeX
  \usepackage{xurl}
  \usepackage{microtype}
  \usepackage[T1]{fontenc}
  \usepackage[utf8]{inputenc}
  \usepackage{textcomp} % provide euro and other symbols
\else % if luatex or xetex
  \usepackage{unicode-math} % this also loads fontspec
  \defaultfontfeatures{Scale=MatchLowercase}
  \defaultfontfeatures[\rmfamily]{Ligatures=TeX,Scale=1}
\fi
\usepackage{lmodern}
\ifPDFTeX\else
  % xetex/luatex font selection
\fi
% Use upquote if available, for straight quotes in verbatim environments
\IfFileExists{upquote.sty}{\usepackage{upquote}}{}
\IfFileExists{microtype.sty}{% use microtype if available
  \usepackage[]{microtype}
  \UseMicrotypeSet[protrusion]{basicmath} % disable protrusion for tt fonts
}{}
\makeatletter
\@ifundefined{KOMAClassName}{% if non-KOMA class
  \IfFileExists{parskip.sty}{%
    \usepackage{parskip}
  }{% else
    \setlength{\parindent}{0pt}
    \setlength{\parskip}{6pt plus 2pt minus 1pt}}
}{% if KOMA class
  \KOMAoptions{parskip=half}}
\makeatother
\usepackage{xcolor}
\usepackage{longtable,booktabs,array}
\usepackage{calc} % for calculating minipage widths
% Correct order of tables after \paragraph or \subparagraph
\usepackage{etoolbox}
\makeatletter
\patchcmd\longtable{\par}{\if@noskipsec\mbox{}\fi\par}{}{}
\makeatother
% Allow footnotes in longtable head/foot
\IfFileExists{footnotehyper.sty}{\usepackage{footnotehyper}}{\usepackage{footnote}}
\makesavenoteenv{longtable}
\setlength{\emergencystretch}{3em} % prevent overfull lines
\providecommand{\tightlist}{%
  \setlength{\itemsep}{0pt}\setlength{\parskip}{0pt}}
\setcounter{secnumdepth}{-\maxdimen} % remove section numbering
\makeatletter
\@ifpackageloaded{subfig}{}{\usepackage{subfig}}
\@ifpackageloaded{caption}{}{\usepackage{caption}}
\captionsetup[subfloat]{margin=0.5em}
\AtBeginDocument{%
\renewcommand*\figurename{Figure}
\renewcommand*\tablename{Table}
}
\AtBeginDocument{%
\renewcommand*\listfigurename{List of Figures}
\renewcommand*\listtablename{List of Tables}
}
\newcounter{pandoccrossref@subfigures@footnote@counter}
\newenvironment{pandoccrossrefsubfigures}{%
\setcounter{pandoccrossref@subfigures@footnote@counter}{0}
\begin{figure}\centering%
\gdef\global@pandoccrossref@subfigures@footnotes{}%
\DeclareRobustCommand{\footnote}[1]{\footnotemark%
\stepcounter{pandoccrossref@subfigures@footnote@counter}%
\ifx\global@pandoccrossref@subfigures@footnotes\empty%
\gdef\global@pandoccrossref@subfigures@footnotes{{##1}}%
\else%
\g@addto@macro\global@pandoccrossref@subfigures@footnotes{, {##1}}%
\fi}}%
{\end{figure}%
\addtocounter{footnote}{-\value{pandoccrossref@subfigures@footnote@counter}}
\@for\f:=\global@pandoccrossref@subfigures@footnotes\do{\stepcounter{footnote}\footnotetext{\f}}%
\gdef\global@pandoccrossref@subfigures@footnotes{}}
\@ifpackageloaded{float}{}{\usepackage{float}}
\floatstyle{ruled}
\@ifundefined{c@chapter}{\newfloat{codelisting}{h}{lop}}{\newfloat{codelisting}{h}{lop}[chapter]}
\floatname{codelisting}{Listing}
\newcommand*\listoflistings{\listof{codelisting}{List of Listings}}
\makeatother
\ifLuaTeX
  \usepackage{selnolig}  % disable illegal ligatures
\fi
\usepackage{bookmark}
\IfFileExists{xurl.sty}{\usepackage{xurl}}{} % add URL line breaks if available
\urlstyle{same}
\hypersetup{
  pdftitle={Designing Incident Response Playbooks},
  hidelinks,
  pdfcreator={LaTeX via pandoc}}

\title{Designing Incident Response Playbooks}
\author{}
\date{2024-02-07}

\begin{document}
\maketitle

\begin{itemize}
\tightlist
\item
  What is my approach to threat modeling? What methodologies do I find
  most useful?
\end{itemize}

{[}{[}Methodologies{]}{]}

\begin{itemize}
\tightlist
\item
  How do I ensure the company controls and processes strike the right
  balance between developer productivity and security?
\end{itemize}

{[}{[}Strategy{]}{]}

\begin{itemize}
\tightlist
\item
  How did I manage building and executing a security strategy from
  scratch?
\end{itemize}

{[}{[}Detailed Approach{]}{]}

\begin{itemize}
\tightlist
\item
  Who are Canonical's key competitors, and how should Canonical set
  about winning?
\end{itemize}

\begin{itemize}
\tightlist
\item
  What would you most want to change about Canonical?
\end{itemize}

\begin{itemize}
\tightlist
\item
  Describe your approach to coaching, mentorship and career development
\end{itemize}

{[}{[}Thoughts on Canonical's Mission{]}{]} {[}{[}Canonical's Key
Competitors{]}{]} {[}{[}Desire to Work for Canonical{]}{]} {[}{[}Changes
Desired in Canonical{]}{]} {[}{[}Excitement About the Role{]}{]}

\begin{itemize}
\tightlist
\item
  An experience in which I successfully coordinated my company's
  security operations with a public agency.
\end{itemize}

{[}{[}Experience Details{]}{]}

\begin{itemize}
\tightlist
\item
  I describe my last experience creating security standards and
  procedures. Which framework did I use and why?
\end{itemize}

{[}{[}Framework Used{]}{]} {[}{[}Reason for Choice{]}{]}

\begin{itemize}
\tightlist
\item
  What approach would you take in designing and implementing incident
  response playbooks? What makes them effective?
\end{itemize}

{[}{[}Design Approach{]}{]} {[}{[}Effectiveness Factors{]}{]}

\end{document}
